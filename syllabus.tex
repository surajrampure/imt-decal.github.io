\PassOptionsToPackage{unicode=true}{hyperref} % options for packages loaded elsewhere
\PassOptionsToPackage{hyphens}{url}
%
\documentclass[]{article}
\usepackage{lmodern}
\usepackage{amssymb,amsmath}
\usepackage{ifxetex,ifluatex}
\usepackage{fixltx2e} % provides \textsubscript
\ifnum 0\ifxetex 1\fi\ifluatex 1\fi=0 % if pdftex
  \usepackage[T1]{fontenc}
  \usepackage[utf8]{inputenc}
  \usepackage{textcomp} % provides euro and other symbols
\else % if luatex or xelatex
  \usepackage{unicode-math}
  \defaultfontfeatures{Ligatures=TeX,Scale=MatchLowercase}
\fi
% use upquote if available, for straight quotes in verbatim environments
\IfFileExists{upquote.sty}{\usepackage{upquote}}{}
% use microtype if available
\IfFileExists{microtype.sty}{%
\usepackage[]{microtype}
\UseMicrotypeSet[protrusion]{basicmath} % disable protrusion for tt fonts
}{}
\IfFileExists{parskip.sty}{%
\usepackage{parskip}
}{% else
\setlength{\parindent}{0pt}
\setlength{\parskip}{6pt plus 2pt minus 1pt}
}
\usepackage{hyperref}
\hypersetup{
            pdftitle={IMT DeCal},
            pdfborder={0 0 0},
            breaklinks=true}
\urlstyle{same}  % don't use monospace font for urls
\setlength{\emergencystretch}{3em}  % prevent overfull lines
\providecommand{\tightlist}{%
  \setlength{\itemsep}{0pt}\setlength{\parskip}{0pt}}
\setcounter{secnumdepth}{0}
% Redefines (sub)paragraphs to behave more like sections
\ifx\paragraph\undefined\else
\let\oldparagraph\paragraph
\renewcommand{\paragraph}[1]{\oldparagraph{#1}\mbox{}}
\fi
\ifx\subparagraph\undefined\else
\let\oldsubparagraph\subparagraph
\renewcommand{\subparagraph}[1]{\oldsubparagraph{#1}\mbox{}}
\fi

% set default figure placement to htbp
\makeatletter
\def\fps@figure{htbp}
\makeatother


\title{IMT DeCal}
\date{}

\begin{document}
\maketitle

\hypertarget{introduction-to-mathematical-thinking}{%
\subsection{Introduction to Mathematical
Thinking}\label{introduction-to-mathematical-thinking}}

\hypertarget{cs-98-xx-uc-berkeley-fall-2018}{%
\subsubsection{CS 98-xx @ UC Berkeley, Fall
2018}\label{cs-98-xx-uc-berkeley-fall-2018}}

Jump to \protect\hyperlink{announcements}{Announcements},
\protect\hyperlink{description}{Course Description},
\protect\hyperlink{logistics}{Logistics},
\protect\hyperlink{schedule}{Content and Schedule}, or
\protect\hyperlink{staff}{Staff}\\
Read the \href{http://book.imt-decal.org}{textbook here}

\hypertarget{announcements}{%
\subsubsection{Announcements}\label{announcements}}

\begin{itemize}
\tightlist
\item
  Please fill out the \textbf{interest form} at
  \href{http://interest.imt-decal.org}{interest.imt-decal.org} to
  indicate that you're interested in taking this course in Fall 2018.
  This is not an application, rather it is for us to gauge interest in
  the course.
\item
  This page will be updated with the latest information once the course
  is approved. Feel free to email {imt-decal@berkeley.edu} with any
  questions!
\end{itemize}

\hypertarget{description}{%
\subsubsection{Course Description}\label{description}}

Berkeley's highly theoretical Computer Science curriculum demands a high
level of mathematical maturity. While those with extracurricular math
experience from high school are familiar with dense notation, complex
mathematical objects, and proof techniques, many students find
foundational courses like CS 70, CS 170, and Math 55 confusing and
inaccessible.

Introduction to Mathematical Thinking bridges the gap.

\textbf{We teach mathematical maturity.} Our curriculum exposes students
to familiar concepts in a more precise, generalized way. By the end of
our course, students will be able to:

\begin{itemize}
\tightlist
\item
  comfortably read mathematical language, including notation,
  definitions and proofs
\item
  concisely and clearly express their ideas differentiate between a good
  proof and a proof with logical gaps
\end{itemize}

As a result, this course will prepare students for higher-level
mathematics courses, such as CS 70 at Berkeley. However, students can
enroll in the course even if they aren't planning on taking these
courses or are not in CS/EECS; these skills and concepts are highly
transferrable.

There are \textbf{no prerequisites} for this course. We're working
really hard to make the material accessible for all backgrounds.

\hypertarget{logistics}{%
\subsubsection{Logistics}\label{logistics}}

All of the following is subject to approval and is tentative.

The course will be offered for 2 units, P/NP. There will be one 1.5h
lecture per week, and one 1.5h discussion per week. There will be weekly
problem sets, which are taken up in discussion section after students
have had a few days to attempt the problems. They are not to be
submitted.

Students' grades for the course are determined on a 100-point scale.

\begin{itemize}
\tightlist
\item
  \textbf{Weekly surveys (5 points x 10 surveys = 50 points)}: Problem
  sets aren't collected, but it is still expected that students attempt
  them. Each week, there will be a form released where students indicate
  how much of the problem set they attempted and how well they
  understood the content. Surveys may also include a short problem that
  should take less than half an hour to complete.
\item
  \textbf{Midterm (20 points)}: A take-home midterm problem set, which
  students will have a week to complete individually.
\item
  \textbf{Final (30 points)}: An in-class final examination at the end
  of the semester.
\end{itemize}

Students will need 75 points to pass the course, which roughly
corresponds to filling out every survey and earning half of the points
on both exams.

\hypertarget{schedule}{%
\subsubsection{Content and Schedule}\label{schedule}}

A textbook is being written specifically for this course. It will be
available for free at
\href{http://book.imt-decal.org}{book.imt-decal.org}. There will also be
lecture videos embedded into the textbook; this is still a work in
progress.

\textbf{Week}

\textbf{Topic}

\textbf{Reading}

\textbf{Problem Set}

0

\textbf{Introduction}\\
Why does this class exist? How to effectively learn mathematics for
problem solving. Overview of the topics covered throughout the course.
Introduce notation.

1

\textbf{Set Theory}\\
Definition of a set. Set operations. Principle of Inclusion-Exclusion.
Begin defining functions.

2

\textbf{Set Theory, Functions}\\
A general definition of functions. Domain, range. Piecewise functions.
Bijections. Construction of natural numbers, integers, rationals, reals
and complex numbers.

3

\textbf{Number Theory}\\
Prime factorization of integers. LCM, GCD. Fundamental theorem of
arithmetic. Introduction to modular arithmetic.

4

\textbf{Proof Techniques}\\
Presentation of various proof techniques, and examples of each. How to
read proofs. Faulty proofs.

5

\textbf{Proof Techniques, Continued}\\
Analysis of proof techniques, continued. Mathematical induction.

6

\textbf{Midterm/Buffer Week}\\

7

\textbf{Series and Sequences}\\
Properties of summation notation. Arithmetic and geometric series.
Telescoping sums and other sums. Generalized sequences.

8

\textbf{Counting}\\
Thinking about the "number of ways" to do something. Permutations and
combinations.

9

\textbf{Counting, Pascal's Triangle}\\
Properties of Pascal's triangle. More combinatorial problems.
Combinatorial proofs.

10

\textbf{Polynomials}\\
Representation of polynomials. Factor and remainder theorems. Existence
of complex roots.

11

\textbf{Combinatorics with Polynomials}\\
Vieta's formulas. Binomial theorem, generalized to multinomials.

12

\textbf{Final}\\
TBD

Staff

\textbf{Instructor}: \href{http://surajrampure.com}{Suraj Rampure}
(suraj.rampure@berkeley.edu)

\textbf{TAs}: TBD

\textbf{Faculty Advisor}:
\href{https://www2.eecs.berkeley.edu/Faculty/Homepages/rao.html}{Satish
Rao}

\end{document}
